%!TEX option = --shell-escape

\documentclass[12pt]{article}

\usepackage[utf8]{inputenc} 
\usepackage{enumitem}
\usepackage[left=3cm, right=3cm, top=3cm, bottom=3cm]{geometry}
\usepackage{minted}
\usepackage{bussproofs}

\setitemize{align=left, topsep = 5pt, parsep = 2pt}

% Math shortcuts
\def\li{\rightarrow}
\def\fax{\forall x.}
\def\fay{\forall y.}
\def\faz{\forall z.}
\def\exx{\exists x.}
\def\exy{\exists y.}
\def\exz{\exists z.}

\title{FMFP Theory a}
\author{isabel.haas@inf.ethz.ch}

\begin{document}
\maketitle


% TODO (maybe):
% Credits: Lectures, Exercises, Max's exercise materials
% Content table
\section{Evaluation strategies}
Haskell: Lazy Evaluation 
\begin{itemize}
    \item argument only evaluaed when no other steps possible
    \item left term is evaluated first
    \item argument made to fit pattern
\end{itemize}
\subsection{Lazy evalutation}
\subsubsection{Sheet 1, Ex. 1}
\begin{minted}{Haskell}
fibLouis :: Int -> Int
fibLouis 0 = 1
fibLouis 1 = 1
fibLouis n = fibLouis (n - 1) + fibLouis (n - 2)
fibEva :: Int -> Int
fibEva n = fst (aux n) where 
    aux 0 = (0, 1)B
    aux n = next (aux (n - 1))
    next (a, b) = (b, a + b)
\end{minted}
\textbf{Lazy Evaluation of fibLouis 4}
\begin{verbatim}
fibLouis 4 =
fibLouis (4-1) + fibLouis (4-2) =
-- most left term is evaluated first
fibLouis 3 + fibLouis (4-2) =
(fibLouis (3-1) + fibLouis (3-2)) + fibLouis (4-2) 
...
((fibLouis 1 + fibLouis (2-2)) + fibLouis (3-2)) + fibLouis (4-2) =
((1 + fibLouis (2-2)) + fibLouis (3-2)) + fibLouis (4-2) =
...
2 + fibLouis 2 =
2 + (fibLouis (2-1) + fibLouis (2-2))
... = 3
\end{verbatim}
\textbf{Lazy Evaluation of fibEva 4}
\begin{verbatim}
fibEva 4 =
fst (aux 4) =
fst (next (aux (4-1))) =
fst (next (aux 3)) =
fst (next (next (aux (3-1)))) =
fst (next (next (aux 2))) =
...
fst (next (next (next (next (0, 1))))) =
fst (next (next (next (1, 0+1)))) =
fst (next (next (0+1, 1+(0+1)))) =
fst (next (1+(0+1), (0+1)+(1+(0+1)))) 
...
fst ((0+1)+(1+(0+1)), (1+(0+1))+((0+1)+(1+(0+1)))) =
(0+1)+(1+(0+1)) =
-- pattern (0+1) is repeated
1 + (1 + 1) =
3
\end{verbatim}

\section{Natural Deduction}
\subsection{Paranthesizing formulas} 
\begin{itemize}
    \item $\land$ binds stronger than $\lor$ stronger than $\li$
    \item $\li$ associates to right; $\land$ and $\lor$ to the left
    \item Negation extend to the right as far as possible: end of line or )
    \item Quantifiers extend to the right as far as possible: end of line or )
\end{itemize}
\begin{tabular}{l l}
    $p \lor q \land \lnot r \li p \lor q$ & $(p \lor (q \land (\lnot r))) \li (p \lor q)$ \\
    $p \li q \lor p \li r$ & $p \li ((q \lor p) \li r$) \\
    $p \land \fax q(x) \lor r$ & $p \land (\fax (q(x) \lor r))$ \\
    $\lnot \fax p(x) \land \fax q(x) \land r(x) \land s$ &  $\lnot( \fax (p(x) \land (\fax ((q(x) \land r(x)) \land s))))$
\end{tabular} 
\subsection{Natural Deduction without quantifiers}
% Max
If you cannot continue, try to add assumptions by using $\lor E$
\subsubsection{Example}
\textbf{Exercise}: $P \ (\lnot A) \land (A \lor B) \li B$ is a tautology \\
First step: Paranthesizing $\Rightarrow$ $P \equiv ((\lnot A) \land (A \lor B)) \li B$ \\
Let $\Gamma \equiv (\lnot A) \land (A \lor B)$

\begin{prooftree}
    \def\ScoreOverhang{1pt}\def\ScoreOverhang{1pt}

    \AxiomC{}
    \RightLabel{$ax$}
    \UnaryInfC{$\Gamma, A \vdash (\lnot A) \land (A \lor B)$}
    \RightLabel{$\land ER$}
    \UnaryInfC{$\Gamma \vdash A \lor B$}
        \AxiomC{}
        \RightLabel{$ax$}
        \UnaryInfC{$\Gamma, A \vdash A$}

        \AxiomC{}
        \RightLabel{$ax$}
        \UnaryInfC{$\Gamma, A \vdash (\lnot A) \land (A \lor B)$}
        \RightLabel{$\land EL$}
        \UnaryInfC{$\Gamma, A \vdash \lnot A$}

        \RightLabel{$\lnot E$}
        \BinaryInfC{$\Gamma, A \vdash B$}
            \AxiomC{}
            \RightLabel{$ax$}
            \UnaryInfC{$\Gamma, B \vdash B$}
        \RightLabel{$\lor E$}
        \TrinaryInfC{$\Gamma \vdash B$}
        \RightLabel{$\li I$}
        \UnaryInfC{ $\vdash (\lnot A) \land (A \lor B)$}
\end{prooftree}

\subsection{Natural Deduction with quantifiers}
If you cannot continue, try to add assumptions by using $\exists E$ \\
Always check side conditions and write it down
\subsubsection{Sheet 2, Ex. 3b}
\textbf{Exercise}: Proof $(\exx P \land Q) \li ((\exx P) \lor (\exx Q)) $ \\
Let $\Gamma \equiv \exx P \land Q, P \land Q$
\begin{prooftree}
    \def\ScoreOverhang{1pt}\def\ScoreOverhang{1pt}

    \AxiomC{}
    \RightLabel{ax}
    \UnaryInfC{$(\exx P \land Q) \vdash (\exx P \land Q)$}
        \AxiomC{}
        \RightLabel{$ax$}
        \UnaryInfC{$\Gamma \vdash P \land Q$}
        \RightLabel{$\land EL$}
        \UnaryInfC{$\Gamma \vdash P$}
        \RightLabel{$\exists I$}
        \UnaryInfC{$\Gamma \vdash \exx P$}

        \AxiomC{}
        \RightLabel{$ax$}
        \UnaryInfC{$\Gamma \vdash P \land Q$}
        \RightLabel{$\land ER$}
        \UnaryInfC{$\Gamma \vdash Q$}
        \RightLabel{$\exists I$}
        \UnaryInfC{$\Gamma \vdash \exx Q$}

        \RightLabel{$\land I$}
        \BinaryInfC{$\Gamma \vdash (\exx P) \lor (\exx Q)$}
    \RightLabel{$\exists E ^{**}$}
    \BinaryInfC{$(\exx P \land Q) \vdash (\exx P) \lor (\exx Q)$}
    \RightLabel{$\li I$}
    \UnaryInfC{$\vdash (\exx P \land Q) \li ((\exx P) \lor (\exx Q))$}
\end{prooftree}

** side condition OK: x not free in $\exx P \land Q$ nor $(\exx P) \lor (\exx Q)$

\section{Binding and $\alpha$-conversion}
\textbf{Bound}: Each occurrence of a variable is bound or free:
A variable occurence x in a formula A is \textbf{bound} if x occurs within a subformula B of A of the form $\exx B$ or $\fax B$.
\textbf{Alpha-conversion}: bound variables can be renamed \\
\textbf{Examples} \\
\begin{tabular}{l l c}
     & & $\alpha$-convertible \\
    $\fax \exy p(x,y)$ & $\fay \exx p(y,x)$ & yes \\
    $\exz \fay p(z, f(y))$ &  $\exy \fay p(y, f(y))$ & no \\
    $(\fax p(x)) \lor (\exx q(x))$ & $(\faz p(z)) \lor (\exy q(y))$ & yes \\
    $p(x) \li \fax p(x)$ & $p(y) \li \fay p(y)$ & no \\
\end{tabular}

\end{document}
