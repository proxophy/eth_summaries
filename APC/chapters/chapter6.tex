\section{Parallel Algorithms}

\begin{tcolorbox}[title=Sections, colback=blue!10, colframe=blue]
    \begin{enumerate}
        \item Warm up: add two numbers
        \item Models and basic concepts
        \item List and trees
        \item Merging and sorting
        \item Connected components
        \item (Bipartite) perfect matching
        \item Modern/Massively parallel computation (MPC)
    \end{enumerate}
\end{tcolorbox}

\subsection{Warm up: add two numbers}

Basic algorithm: carry-ripple \\
Problematic part: preparation of carry bits \\
Idea: kill/propagate/generate bit, compute that in parallel $\Rightarrow$ use binary tree

\subsection{Models and basic concepts}

Two models: logic circuits and PRAM (parallel random access machines)

\subsubsection{Circuits}
Circuits: Boolean circuits with AND/OR/NOT gates, connected by wires 

$NC(i)$: set of all decision problems that can be decided in boolean circuit with $poly(n)$ gates of at most two imputs and depth of at most $O(\log^i n)$ depth \\
$AC(i)$: set of all decision problems that can be decided in boolean circuit with $poly(n)$ gates of potentially unbound fan-in and depth of at most $O(\log^i n)$ depth

\begin{lemma}
    For any $k$, we have $NC(k) \subseteq AC(k) \subseteq NC(k+1)$   
\end{lemma}
Def: $NC = \cup_i NC(i) = \cup_i AC(i)$

\subsubsection{Parallel random access machines (PRAM)}

$p$ number of RAM processors, each with its own local registers and access to a global (shared) memory \\
Four variations: Exclusive/Concurrent Read Eclusive/Concurrent Write

\begin{lemma}
    For any $k$, we have $CRCW(k) \subseteq EREW(k+1)$
\end{lemma}
We also have $NC=\cup_k EREW(k)$: PRAM can simulate circuits and vice versa

\subsubsection{Some basic problems}

\textbf{Parallel Prefix}: Task: Given array $A$ of length $n$, compute $B$ with $B[j]= \sum_{i=1}^j A[i]$, Can be parallized with binary tree idea

\textbf{List ranking}: Task: Given linked list by its content array $c[1..n]$ and successor pointer array $s[1..n]$, get all suffix sums $\Rightarrow$ Use algorithm based on idea of \textbf{pointer jumping} \\
List ranking with pointer jumping: $\log n$ iterations, each with two steps:
\begin{enumerate}
    \item In parallel, for each $i\in \{1,\dots, n\}$, set $c(i) = c(i) + c(s(i))$
    \item In parallel, for each $i\in \{1,\dots, n\}$, set $s(i)=s(s(i))$
\end{enumerate}

\begin{lemma}
    At the end of the above process, for each $i\in \{1,\dots, n\}$, c(i) is equal to its suffix sum
\end{lemma}

\subsubsection{Work-efficient parallel algorithms}

Computational process can be viewed as a sequence of rounds, each rounds consits of number of computations indepentend of each other, computable in parallel 

Total number of rounds: \textbf{depth} of computation; Summation of number of computaions: total \textbf{work} 

Primary goal: depth small as possible, second goal: small total work

\begin{theorem}[Brent's principle]
    If an algorithm does $x$ computations in total and has depth $t$, then using $p$ processors, the algorithm can be run in $x/p + t$ time
\end{theorem}

\textbf{Work-efficient} algorithm: total amount of work of parallel algorithm is proportional to amount of work in sequential case.

\section{Lists and trees}
\subsection{List ranking}
Goal: Obtain algorithm with $O(\log n)$ depth and $O(n)$ total work


